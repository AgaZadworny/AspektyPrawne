\documentclass{sprawozdanie-agh}

\usepackage[utf8]{inputenc}
\usepackage{listings}
\usepackage{pdfpages}
\usepackage{float}
\usepackage{anyfontsize}
\usepackage{graphicx}
 
\makeatletter 

\begin{document}   

	\przedmiot{Aspekty prawne i organizacja przedsiębiorstwa}
	\tytul{„Wypożyczalnia rowerów”}
	\podtytul{Jednoosobowa działalność gospodarcza}
	\kierunek{Informatyka, III rok, 2018/2019}
	\autor{Agnieszka Zadworny, Maciej Bielech, Tomasz Pęcak, Piotr Morawiecki}
	\data{Kraków, 13 grudnia 2018}

	\stronatytulowa{}

	\section{Informacje wstępne}

		\subsection{Cel i okoliczności powstania firmy}

		Przedsiębiorca powinien nie tylko dostrzec szansę zyskownej działalności, ale także podjąć decyzję i wprowadzić ją w życie z sukcesem ekonomicznym.
		
		Dokładna analiza wszystkich etapów przedsięwzięcia daje wskazówki dotyczące opłacalności danego planu. Istnieją metody i procedury dostępne każdej osobie zajmującej się biznesem, które pozwalają przeanalizować wszystkie sytuacje w biznesie. Decyzje dotyczące „robienia interesów” nie powinny być dokonywane pod wpływem impulsu. Powinny być poprzedzane dogłębną analizą wszystkich aspektów przedsięwzięcia. Takie badania nazywane są analizą wykonalności projektu. Należy jednak pamiętać, że nawet najbardziej systematyczna analiza wykonalności projektu nie gwarantuje powodzenia. Nawet prosta sytuacja staje się złożona, jeśli analiza przekroczy granice rozsądku. Można wpaść w pułapkę stosowania dokładnych analiz jako jedynych wskazówek w procesie dokonywania decyzji. Analizy te, o czym należy pamiętać, opierają się również na założeniach i ocenach szacunkowych. Należy je traktować jako jedną z wielu wskazówek przy podejmowaniu decyzji.
		
		Istnieje wiele przykładów, które dowodzą, że podejmowanie decyzji na podstawie intuicji zaowocowało wieloma nieprawdopodobnie udanymi przedsięwzięciami. Decyzja przedsiębiorcy oparta na wewnętrznym przekonaniu jest często podświadomie kierowana doświadczeniem i talentem do biznesu, a zatem taka decyzja wcale nie jest przypadkowa. Szczególnie dotyczy to nowych pionierskich dziedzin biznesu, gdzie nie istnieją wypróbowane metody ani procedury.
		
		Walory turystyczne powiatu przemyskiego mogą stać się inspiracją dla bezrobotnych z tego regionu do uruchomienia własnej działalności gospodarczej.
		
		Istnieje wiele możliwości i sposobów stworzenia nowych miejsc pracy dla przedsiębiorczych bezrobotnych z tego regionu. Można wymienić kilka pomysłów na wykorzystanie walorów turystycznych regionu, które wydają się ciekawe, atrakcyjne i mają szansę na powodzenie takich przedsięwzięć.
		
		\subsubsection{Charakterystyka projektowanego przedsięwzięcia}
		Jednym z nich mogłoby być stworzenie ścieżek rowerowych w powiecie przemyskim po rezerwatach przyrody ziemi przemyskiej i innych atrakcjach turystycznych. Powstanie takich ścieżek cieszyłoby się dużą atrakcyjnością dla mieszkańców Przemyśla i okolic, chcących w sposób czynny spędzić czas wolny od pracy, podziwiać piękno przyrody. Tacy weekendowi turyści mogliby zatrzymywać się przy atrakcyjnych miejscach w celu ich zwiedzania. Dla nich potrzebne byłoby stworzenie małej gastronomii, barów szybkiej obsługi, itp. Propozycja skierowania przedsiębiorczości bezrobotnych w tym kierunku wydaje się być nie tylko atrakcyjna, ale ma szansę powodzenia dla bezrobotnych poszukujących możliwości przedsiębiorczego działania, a tym samym stworzenia nowych miejsc pracy.
		
		Aby zrealizować ten plan, należałoby w pierwszej kolejności skierować przedsiębiorczość bezrobotnych na rozreklamowanie walorów turystycznych Przemyśla i regionu wśród mieszkańców. Takim sposobem reklamy mogłoby stać się rozsyłanie darmowych widokówek do biur turystycznych oraz reklama zabytków i miejsc wartych odwiedzenia w lokalnej telewizji miast partnerskich miasta Przemyśla.
		
		\subsubsection{Zasięg terytorialny projektowanej ścieżki rowerowej}
		Zaproponowano utworzenie dwóch ścieżek rowerowych, których trasa wiodłaby przez lasy przemyskie, obok jezior i atarakcji turystycznych:
		\begin{itemize}
			\item jedna ze ścieżek byłaby ścieżką krótszą, łatwiejszą, z której korzystałyby rodziny z dziećmi lub osoby w starszym wieku, lubiące czynną formę wypoczynku lub też osoby, które nie mogą poświęcić sporo czasu na tę formę wypoczynku,
			\item druga ścieżka byłaby ścieżką do pasjonatów jazdy rowerowej, którzy lubią spędzać wolny czas z dala od miejskiego zgiełku, blisko przyrody (las, woda), wędkując, kąpiąc się w akwenach wodnych, chcących skorzystać z pola namiotowego w okolicy tych akwenów lub spacerując po lesie, zbierając grzyby, korzystając z baru szybkiej obsługi - mogliby tutaj odpocząć po trudach jazdy rowerowej, zjeść posiłek, wypić napoje chłodzące lub zjeść lody.
		\end{itemize}
	
		Trasa miałaby charakter ciągły. Nie budowałoby się nowej asfaltowej ścieżki rowerowej, lecz trasy te (krótsza i dłuższa) przebiegałyby z wykorzystaniem istniejących dróg asfaltowych lub utwardzonych.
	 
		W Wyszatycach wydzierżawiono by teren o powierzchni 400 m2 na potrzeby użytkowników ścieżek rowerowych (zostawienie samochodów, postawienie budki na przechowanie  rowerów, które uczestnicy tych tras mogliby tu wypożyczać).
	
		\subsubsection{Opis przedsięwzięcia według pomysłu autorki niniejszej pracy}
		
		Projekt dotyczy otwarcia wypożyczalni rowerów na terenie regionu przemyskiego. Wypożyczalnia będzie funkcjonowała tuż przy nowo powstałej sieci ścieżek rowerowych. Wybór wynika z zapotrzebowania na tego rodzaju aktywny wypoczynek dla osób miejscowych i przybyłych z zewnątrz. Chcę aby moja oferta była skierowana do ludzi na każdej płaszczyźnie wiekowej, dla grup osób indywidualnych, jak i całych rodzin.
		
		\subsubsection{Profil przedsiębiorstwa}
		Informacje o przedsiębiorstwie (nazwa, siedziba, status, bank):\\
		„Rowerek - Anna Kowalska”,\\
		ul. Kwiatowa 2/4\\
		37-722 Wyszatyce.\\
		Strona internetowa: www.rowerek.pl\\
		Jednoosobowa działalność gospodarcza\\
		Bank nr konta: Bank PEKAO S.A. I O/Przemyśl\\
		Ul. Fiołkowa 37\\
		Nr konta 10000381240279918125200012.\\
		Nr identyfikacji podatnika podatku VAT – 123-456-78-90.
		
		\subsection{Działalność przedsiębiorstwa}

		\subsubsection{Branża}
		Branża rekreacji i sportu

		\subsubsection{Rynek}
		Rynek związany z kolarstwem, wypoczynkiem i turystyką

		\subsubsection{Docelowa grupa klientów}
		Oferta firmy „Rowerek” skierowana jest do szerokiego grona konsumentów, jakimi są młodzi i starsi ludzie preferujący wypoczynek na świeżym powietrzu, goście z krajów UE, ale także mieszkańcy gminy. Specyficzna atmosfera jaką chcemy stworzyć ma zaznajomić i przybliżyć kulturę i zwyczaje naszego regionu.

		\subsubsection{Przedmiot działalności}
		\begin{itemize}
			\item Pozostała działalność rozrywkowa i rekreacyjna
			\item Pozostała działalność związana ze sportem
		\end{itemize}

		\subsection{Forma prawna przedsiębiorstwa - jednoosobowa działalność gospodarcza}

		Jednoosobowa działalność gospodarcza (działalność indywidualna) jest najprostszym rodzajem firmy. Nazywana jest często samozatrudnieniem, może być założona wyłącznie przez osobę fizyczną. Dzięki temu, że jej rejestracja i prowadzenie nie są skomplikowane, jest najbardziej odpowiednią formą działalności dla drobnego biznesu.

		\subsubsection{Cechy jednoosobowej działalności}
		Osoba fizyczna prowadząca działalność posiada zdolność prawną, zdolność do czynności prawnych oraz zdolność sądową i procesową. Do prowadzenia jednoosobowej działalności nie trzeba sporządzać żadnych umów. Nie jest wymagany minimalny kapitał.

		\subsubsection{Odpowiedzialność za zobowiązania}
		Przedsiębiorca odpowiada całym swoim majątkiem za zobowiązania powstałe w wyniku prowadzenia działalności. Odpowiedzialność ta rozciąga się również na małżonka (z wyłączeniem jego majątku osobistego). Oznacza to, że prywatny majątek przedsiębiorcy i ten, wykorzystywany do prowadzenia działalności, są przez wierzycieli traktowane jednakowo. Jeżeli działalność nie wypracuje zysku, wszystkie zobowiązania (np. składki do ZUS, wynagrodzenia dla pracowników i współpracowników, rachunki, kary umowne, raty kredytu) muszą zostać pokryte z prywatnego majątku. Wszystkie zyski osiągnięte z prowadzonej działalności od razu zwiększają prywatny majątek przedsiębiorcy.

		Przedsiębiorca ma również dodatkowe obowiązki informacyjne. Jeżeli wysyła korespondencję do innej osoby, musi wskazać nazwę swojej firmy, NIP i adres.

		\subsubsection{Reprezentowanie firmy}
		Jednoosobowy przedsiębiorca ma wyłączne prawo do reprezentowania swojej działalności. Tylko on może zawierać umowy z dostawcami i odbiorcami, tylko on ma obowiązek wpłacać zaliczki na podatek, składać roczną deklarację podatkową i rozliczać się z ZUS. Oczywiście część tych rzeczy może zlecić innym osobom na podstawie udzielonego pełnomocnictwa, jednak zawsze działanie jego pełnomocników będzie miało wpływ na jego majątek prywatny.

		\subsubsection{Nazwa działalności}
		Każda działalność musi mieć swoją nazwę. W przypadku wpisu do CEIDG, nazwą jest co najmniej imię i nazwisko.
		
		\subsection{Przeprowadzone postępowanie przygotowawcze w celu założenia przedsiębiorstwa}
		Ważną rzeczą jest to, że firma „Rowerek” powstaje na nowo otwartych możliwościach, gdzie do tej pory nie było tego typu propozycji wypoczynku. Wynika z tego, że konkurencja jest niewielka i co za tym idzie, możliwość pozyskania klientów jest duża. Miejsce rozdawania ulotek, udzielania informacji będą dokładnie rozpatrzone. Na pewno jednym ze sposobów będzie skorzystanie z lokalnej i regionalnej prasy: "Życie podkarpackie", Biuletyn Miejski miasta Przemyśl oraz dzienniki Super Nowości i Kurier Powiatowy.
		
		Ulotki oraz logo firmy zaprojektuje właścicielka Anna Kowalska.

		\subsubsection{Możliwości i szanse}
		Ta oferta skierowana jest do całego społeczeństwa, zarówno zamożnych jak i ubogich klientów, miejscowych i zagranicznych. Dajemy możliwość oderwania się od zgiełku zatłoczonych miast i warkotu jeżdżących samochodów. To czysty kontakt z naturą wśród pachnących łąk i lasów.

		\subsubsection{Podstawowe zagrożenia}
		Brak infrastruktury wzdłuż ścieżek rowerowych tj. kiosków z napojami chłodzącymi, z regionalną kuchnią, brak drogowskazów określających kierunek jazdy, punktów informacyjnych o miejscowych zabytkach, pamiątkach przyrody, czy ciekawych wydarzeniach. To spowoduje skrócenie czasu obcowania z przyrodą, gdyż głód i zmęczenie pokieruje klientów w drogę powrotną.

		\subsubsection{Analiza otoczenia konkurencyjnego}
		Z przeprowadzonej ankiety wynika, że wiele osób wyjeżdża na dłuższe i krótsze weekendy do pobliskiego Rezerwatu Przyrody Starzawa nad zalew lub na pobliską Ukrainę tylko dlatego, że nie ma na miejscu żadnej możliwości skorzystania z wypoczynku na świeżym powietrzu, a więc wielu z nich jest zainteresowanych taką formą wypoczynku, jaką będzie się zajmowała firma „Rowerek”.

		\subsubsection{Analiza SWOT}
		Na podstawie przeprowadzonych badań zostały określone mocne i słabe strony działalności oraz szanse i zagrożenia w otoczeniu firmy, będące podstawą do poznania charakterystycznych cech firmy, jej możliwości rozwoju oraz pozycji w stosunku do konkurencji w sektorze.
		\subsubsection{Mocne strony}

		\begin{itemize}
			\item Jedyna w swoim rodzaju (brak konkurencji).
			\item Skierowana do zamożnych i mniej zamożnych klientów.
			\item Dostosowana do potrzeb osób starszych i rodzin z małymi dziećmi.
			\item Wysoki poziom kapitału własnego.
			\item Łatwy dostęp do dostawców regionalnych(pamiątki, potrawy).
			\item Samozatrudnienie.
			\item Wpisanie się w ogólny trend promocji zdrowego trybu życia
		\end{itemize}
		\subsubsection{Słabe strony}

		\begin{itemize}
			\item Brak infrastruktury wzdłuż ścieżek rowerowych.
			\item Niedostateczny przepływ informacji dotyczący potrzeb klientów.
			\item Brak jednolitej strategii rozwoju regionu(spójności działań ze strony władz miasta i gmin powiatu Przemyśl).
			\item Sezonowość.
			\item Warunki pogodowe.
		\end{itemize}
		\subsubsection{Szanse}

		\begin{itemize}
			\item Napływ obcokrajowców i turystów z różnych stron Polski.
			\item Korzystanie z funduszy strukturalnych przeznaczonych na rozwój regionu.
			\item Wzrost zamożności społeczeństwa.
			\item Możliwość rozwoju spółki – wzrost, zapotrzebowania na imprezy.
			\item Rozwój sieci hotelarsko – gastronomicznej.
			\item Duża grupa lojalnych klientów.
		\end{itemize}
		\subsubsection{Zagrożenia}

		\begin{itemize}
			\item Rosnąca liczba poważnych, aktywnie działających na rynku konkurentów.
			\item Coraz bardziej atrakcyjna oferta firm konkurencyjnych.
			\item Niż demograficzny.
			\item Spadek realnej wartości złotówki.
		\end{itemize}
		\subsubsection{Kapitał i cennik}

		Właściciel spółki posiada kapitał własny z oszczędności.
		Dodatkowo firma „Rowerek - Anna Kowalska'' zaciągnęła kredyt w wysokości 15 000 zł. 
		
		Całkowity kapitał przeznaczony zostanie na:
		\begin{itemize}
			\item zakup 10 szt. rowerów	- 10 000  zł
			\item dzierżawa parkingu 400 m2 z 2 kabinami toaletowymi (miesięcznie) - 370 zł
			\item wiata do przechowywania rowerów - 900 zł
			\item czynności formalno-prawne	- 300 zł
			\item reklama lokalu „Rowerek'' - 300 zł (ulotki, „blackboardy'')
			\item przyczepa campingowa (dzierżawa)	-  950 zł
			\item wyposażenie (chłodziarka, czajnik	elektryczny, grill, kuchenka) - 3000 zł
			\item 4 stoliki i 16 krzeseł - 1200 zł
			\item zakup pierwszej partii artykułów spożywczych - 700 zł
			\item postawienie budki - 2 600 zł
		\end{itemize}
		
		Koszty razem - 20 320 zł.
		\subsubsection{Przychody}

		\begin{itemize}
			\item pozostawienie samochodu na parkingu (dzień) - 10,00 zł
			\item wynajęcie roweru (dorosły, godzina) - 6,00 zł
			\item wynajęcie roweru (dorosły, dzień) - 25,00 zł
			\item wynajęcie roweru (dziecko, godzina) - 4,00 zł
			\item wynajęcie roweru (dziecko, dzień) - 15,00 zł
			\item umycie samochodu - 15,00 zł
			\item naprawa roweru - 50 zł + koszt części
			\item konserwacja roweru - 20 zł
			\item sprzedaż map - 8,00 zł
			\item sprzedaż informatorów - 5,00 zł
			\item posiłki
				\begin{itemize}
					\item zupy - 5 zł
					\item bigos - 7 zł
					\item pierogi - 10 zł
					\item gofry z owocami i bitą śmietaną - 8 zł
					\item zapiekanki - 9 zł
					\item frytki - 6 zł
					\item hamburgery - 8 zł
					\item herbata - 4 zł
					\item kawa - 5 zł
					\item soki - 3 zł
					\item pepsi - 4 zł
					\item lody - gałka - 3 zł
				\end{itemize}
		\end{itemize}
		
		\subsubsection{Wynajęcie parkingu}

		„Rowerek” w celu otwarcia wypożyczalni rowerów Firma wynajmie powierzchnię asfaltową – 400 m2. Umowa wynajmu pomiędzy firmą „Rowerek” a właścicielem działki w Wyszatycach przy ul. Sportowej 18
		
		\subsubsection{Budka}

		Budka będzie służyła do sporządzania posiłków regionalnych, kanapek, grillowanych kiełbasek, parzenia kawy, herbaty oraz sprzedaży napojów chłodzących.
		
		Przewidywane umiejscowienie budki – na szlaku ścieżki rowerowej w połowie trasy rowerowej do Leszna. Posiłki będzie można zjeść przy stolikach pod zadaszeniem w chłodnym i czystym miejscu w otoczeniu przyrody wśród śpiewających ptaków.
		
		Zastawy obiadowe oraz sztućce będą przyozdobione ludowymi motywami ustawiane na serwetach haftowanych przez tutejsze gospodynie.
		
		\subsubsection{Produkt}

		Firma „Rowerek” oferować będzie:
		\begin{itemize}
			\item dania gorące (wyszczególnione w karcie dań),
			\item napoje bezalkoholowe: kawa, herbata, soki, napoje,
			\item lody z owocami.
		\end{itemize}
		Towary długoterminowe zakupywane będą w hurtowniach, a krótkoterminowe u lokalnych dostawców.

	\section{Rejestracja firmy}

		\subsection{Instytucje do których należy się udać (i co tam załatwić)}
		
		\begin{itemize}
			\item Bank \\
			Pomimo, że nie jest to wymagane, zaleca się założyć firmowe konto bankowe.
			\item Urząd gminy \\
			W celu rejestracji firmy należy udać się osobiście do urzędu gminy w celu złożenia odpowiedniego wniosku CEIDG-1, alternatywnie można wysłać ten wniosek elektronicznie. 
			\item ZUS \\
			W celu uzyskania ubezpiecznia należy w terminie do 7 dni od daty rozpoczęcia działalności zgłosić siebie do ubezpieczeń i złożyć w Zakładzie Ubezpieczeń Społecznych formularza ZUA lub ZZA. 
			\item Ze względu na roczne obroty nie przekraczające 200 000 zł korzystamy z możliwości zwolnienia z rejestracji jako czynny podatnik VAT.
		\end{itemize}
		
		\subsection{Jakie druki wypełnić w tych instytucjach?}

		\begin{itemize}
			\item Urząd gminy
			\item CEIDG-1 - formularz, który służy głównie do rejestracji nowo założonej działalności gospodarczej.
		\end{itemize}
		Składając CEIDG-1 wnioskujemy dodatkowo o:
		\begin{itemize}
			\item uzyskanie wpisu w rejestrze REGON
			\item zgłoszenie identyfikacyjne do naczelnika urzędu skarbowego, w celu otrzymania NIP 
			\item oświadczenie o wyborze formy opodatkowania
		\end{itemize}

		\subsection{Jakie dokumenty należy posiadać?}
		Złożenie wniosku CEIDG-1 wymaga jedynie okazania dokumentu potwierdzającego tożsamość wnioskodawcy. Przy zakładniu firmy onilne można skorzystać z podpisu elektroniczego.
	\section{Formy opodatkowania przedsiębiorstwa}

		\subsection{Dopuszczalne formy opodatkowania naszej firmy w Polsce}
		\begin{itemize}
			\item PIT progresywny (na zasadach ogólnych)
			\item Podatek liniowy 
			\item Ryczałt od przychodów ewidencjonowanych (RPE)
			\item Karta podatkowa
		\end{itemize}
		\subsection{Wybrana forma opodatkowania: charakterystyka i dlaczego}
		Wybraliśmy podatek na zasadach na zasadach ogólnych ze względu na szacowane niskie dochody (nie przekraczające 85 528 zł) oraz możliwość skorzystania z różnego rodzaju ulg podatkowych.
	\section{Zatrudnianie pracowników}

		\subsection{Procedura zatrudnienia pracownika, wymagane dokumenty i czynności}

		\subsubsection{Zalety umowy zlecenie}
		\begin{itemize}
			\item jest tańszym sposobem zatrudnienia niż umowa o pracę
			\item można wypowiedzieć umowę w każdym czasie (pod warunkiem, że w umowie nie określono okresu wypowiedzenia)
			\item gdy pracownik rozwiązuje umowę przed wykonaniem zlecenia określonego w umowie pracodawca może domagać się zwrotu wydatków, które poniósł w związku z realizacją umowy
			\item brak ciągłego nadzoru nad pracownikiem
			\item jeżeli w umowie określono okres wypowiedzenia to pracownik nie może wypowiedzieć umowy z dnia na dzień
		\end{itemize}
		
		\subsubsection{Wymagane dokumenty}
		Umowa zlecenia

		\subsubsection{Plan zatrudnienia pracowników}

		\begin{itemize}
		\item Właścicielka - księgowość, kierownictwo. 
		\item Serwisant - zajmie się sprawami napraw, parkingu i wypożyczaniem rowerów.
		\item Kucharka - będzie przygotowywała posiłki, sprzedawała herbatę, kawę, napoje chłodzące, lody.
		\end{itemize}
	
		\subsection{PLAN FINANSOWY}
		\subsubsection{Koszty}
		Miesięczny koszt zatrudnienia (brutto) oraz składki ZUS:
		\begin{itemize}
			\item kucharz x 1 na umowę zlecenie: 15 zł za godzinę - 120 godzin w miesiącu (piątek, sobota, niedziela 10:00 - 20:00) - 1 800 zł brutto miesięcznie, a w okresie wakacyjnym - (środa do niedzieli 10:00 - 20:00) - 200 godzin w miesiącu - 3 000 zł brutto miesięcznie,
			\item serwisant x 1 na umowę zlecenie: 14 zł za godzinę - 120 godzin w miesiącu (piątek, sobota, niedziela 10:00 - 20:00) - 1 680 zł brutto miesięcznie, a w okresie wakacyjnym - (środa do niedzieli 10:00 - 20:00) - 200 godzin w miesiącu - 2 800 zł brutto miesięcznie
			\item właściciel: 200,16 zł miesięcznie składki ZUS dzięki skorzystaniu z opłacania "małego ZUS",
		\end{itemize}
		Miesięczny koszt ogółem: 
		\begin{itemize}
			\item rok szkolny: 3 680,16 zł
			\item wakacje: 6 000,16 zł
		\end{itemize}

	\section{Wnioski} 

		\subsection{Zalety i wady wybranej formy prawnej przedsiębiorstwa}
 
		Zalety:

		\begin{itemize}
			\item nie potrzeba kapitału zakładowego
			\item nie trzeba prowadzić pełnej księgowości, można prowadzić księgowość uproszczoną, o ile obroty nie przekraczają równowartości 1 200 000 euro
			\item jest większa możliwość korzystania z ulg i dotacji; nowy przedsiębiorca może np. opłacać składki ZUS w niższej, preferencyjnej stawce przez pierwsze 24 miesiące prowadzenia firmy
			\item masz duże szanse na otrzymanie dotacji unijnych oraz dofinansowania z urzędu pracy
		\end{itemize}
		Wady:

		\begin{itemize}
			\item odpowiada się osobiście za wszystkie długi firmy całym swoim majątkiem, odpowiedzialność rozciąga się również na współmałżonka - z wyłączeniem jego majątku odrębnego;
			\item jednoosobowa działalność gospodarcza jest zarejestrowana na konkretną osobę - nie podlega dziedziczeniu ani nie można jej sprzedać.
		\end{itemize}


		\subsection{Trudności napotkane w trakcie rejestracji firmy}

		\begin{itemize}
			\item Wybór odpowiedniego rodzaju odpodatkowania,
			\item Poprawne wypełnienie formularza CEIDG-1,
			\item Długopis mi się wypisał.
		\end{itemize}

		\subsection{Udogodnienia napotkane w trakcie rejestracji firmy}

		\begin{itemize}
			\item Możliwość złożenia wniosku przez internet,
			\item Możliwość złożenia wniosku przez telefon dla osób niepełnosprawnych,
			\item Podpowiadanie miejscowości i powiatów.
		\end{itemize}

		\subsection{Co przyspieszyłoby proces rejestracji firmy?}

		\begin{itemize}
			\item Lepsza obsługa i mniejsze kolejki w urzędach,
			\item Czas oczekiwania, mógłby być krótszy,
			\item Dłuższy czas trwania sesji na stronie internetowej (podczas wyjścia na kawę uzupełnione dane we wniosku zostały usunięte)
		\end{itemize}

		\subsection{Inne wnioski}
		
		Opracowany biznesplan jest szansą dla rozwoju firmy „Rowerek”, a jego realizacja pozwoli właścicielce spełnić marzenie o własnej działalności gospodarczej. Z analizy rachunku zysków i strat wynika, że uruchomienie takiej działalności pozwoli na zapłacenie bieżących rachunków, opłacenie składek na ubezpieczenie społeczne, wypłacenie wynagrodzenia oraz spłaty miesięcznej raty zaciągniętego kredytu. Prognozy finansowe są obiecujące dla rozwoju firmy, a osiągnięte w przyszłości zyski zapewnią mocną pozycję przedsiębiorstwa w tej branży i pozwolą na inwestowanie w jego rozwój.
		
		Przedstawiony zarys pomysłu na uruchomienie jedoosobowej działalności gospodarczej z wykorzystaniem walorów turystycznych i kulturowych Przemyśla i okolic, jest przykładem innowacyjnego działania, przejawem poszukiwania pomysłu na samozatrudnienie. Jest to niewątpliwie jeden z pomysłów działalności gospodarczej na zasadach samozatrudnienia, a nie oczekiwania tylko na pracę u innych pracodawców.

\end{document}