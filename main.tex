\documentclass{sprawozdanie-agh}

\usepackage[utf8]{inputenc}
\usepackage{listings}
\usepackage{pdfpages}
\usepackage{float}
\usepackage{anyfontsize}
\usepackage{graphicx}
 
\makeatletter 

\begin{document}   

	\przedmiot{Aspekty prawne i organizacja przedsiębiorstwa}
	\tytul{„Wypożyczalnia rowerów”}
	\podtytul{Jednoosobowa działalność gospodarcza}
	\kierunek{Informatyka, III rok, 2018/2019}
	\autor{Agnieszka Zadworny, Maciej Bielech, Tomasz Pęcak, Piotr Morawiecki}
	\data{Kraków, 13 grudnia 2018}

	\stronatytulowa{}

	\section{Informacje wstępne}

		\subsection{Cel i okoliczności powstania firmy}

		Przedsiębiorca powinien nie tylko dostrzec szansę zyskownej działalności, ale także podjąć decyzję i wprowadzić ją w życie z sukcesem ekonomicznym.
		
		Dokładna analiza wszystkich etapów przedsięwzięcia daje wskazówki dotyczące opłacalności danego planu. Istnieją metody i procedury dostępne każdej osobie zajmującej się biznesem, które pozwalają zanalizować wszystkie sytuacje w biznesie. Decyzje dotyczące „robienia interesów” nie powinny być dokonywane pod wpływem impulsu. Powinny być poprzedzane dogłębną analizą wszystkich aspektów przedsięwzięcia. Takie badania nazywane są analizą wykonalności projektu. Należy jednak pamiętać, że nawet najbardziej systematyczna analiza wykonalności projektu nie gwarantuje powodzenia. Nawet prosta sytuacja staje się złożona, jeśli analiza przekroczy granice rozsądku. Można wpaść w pułapkę stosowania dokładnych analiz jako jedynych wskazówek w procesie dokonywania decyzji. Analizy te, o czym należy pamiętać, opierają się również na założeniach i ocenach szacunkowych. Należy je traktować jako jedną z wielu wskazówek przy podejmowaniu decyzji.
		
		Istnieje wiele przykładów, które dowodzą, że podejmowanie decyzji na podstawie intuicji zaowocowało wieloma nieprawdopodobnie udanymi przedsięwzięciami. Decyzja przedsiębiorcy oparta na wewnętrznym przekonaniu jest często podświadomie kierowana doświadczeniem i talentem do biznesu, a zatem taka decyzja wcale nie jest przypadkowa. Szczególnie dotyczy to nowych pionierskich dziedzin biznesu, gdzie nie istnieją wypróbowane metody ani procedury.
		
		Walory turystyczne powiatu przemyskiego mogą stać się inspiracją dla bezrobotnych z tego regionu do uruchomienia własnej działalności gospodarczej.
		
		Istnieje wiele możliwości i sposobów stworzenia nowych miejsc pracy dla przedsiębiorczych bezrobotnych z tego regionu. Można wymienić kilka pomysłów na wykorzystanie walorów turystycznych regionu, które wydają się ciekawe, atrakcyjne i mają szansę na powodzenie takich przedsięwzięć.
		
		\textbf{Charakterystyka projektowanego przedsięwzięcia}\\
		Jednym z nich mogłoby być stworzenie ścieżek rowerowych w powiecie przemyskim po rezerwatach przyrody ziemi przemyskiej. Powstanie takich ścieżek cieszyłoby się dużą atrakcyjnością dla mieszkańców Przemyśla i okolic, chcących w sposób czynny spędzić czas wolny od pracy, podziwiać piękno przyrody. Tacy weekendowi turyści mogliby zatrzymywać się przy atrakcyjnych miejscach w celu ich zwiedzania. Dla nich potrzebne byłoby stworzenie małej gastronomii, barów szybkiej obsługi, itp. Propozycja skierowania przedsiębiorczości bezrobotnych w tym kierunku wydaje się być nie tylko atrakcyjna, ale ma szansę powodzenia dla bezrobotnych poszukujących możliwości przedsiębiorczego działania, a tym samym stworzenia nowych miejsc pracy.
		
		Aby zrealizować ten plan, należałoby w pierwszej kolejności skierować przedsiębiorczość ambitnych bezrobotnych na rozreklamowanie walorów turystycznych Przemyśla i regionu wśród mieszkańców. Takim sposobem reklamy mogłoby stać się rozsyłanie darmowych widokówek do biur turystycznych oraz reklama zabytków i miejsc wartych odwiedzenia w lokalnej telewizji miast partnerskich miasta Przemyśla.
		
		\textbf{Zasięg terytorialny projektowanej ścieżki rowerowej}\\
		Zaproponowano utworzenie dwóch ścieżek rowerowych, których trasa wiodłaby przez lasy przemyskie i okolic:
		\begin{itemize}
			\item jedna ze ścieżek byłaby ścieżką krótszą, łatwiejszą, z której korzystałyby rodziny z dziećmi lub osoby w starszym wieku, lubiące czynną formę wypoczynku lub też osoby, które nie mogą poświęcić sporo czasu na tę formę wypoczynku,
			\item druga ścieżka byłaby ścieżką do pasjonatów jazdy rowerowej, którzy lubią spędzać wolny czas z dala od miejskiego zgiełku, blisko przyrody (las, woda), wędkując, kąpiąc się w akwenach wodnych, chcących skorzystać z pola namiotowego w okolicy tych akwenów lub spacerując po lesie, zbierając grzyby, korzystając z baru szybkiej obsługi - mogliby tutaj odpocząć po trudach jazdy rowerowej, zjeść posiłek, wypić napoje chłodzące lub zjeść lody.
		\end{itemize}
	
		Trasa miałaby charakter ciągły. Nie budowałoby się nowej asfaltowej ścieżki rowerowej, lecz trasy te (krótsza i dłuższa) przebiegałyby z wykorzystaniem istniejących dróg asfaltowych lub utwardzonych.
	
		W Wyszatycach wydzierżawiono by teren o powierzchni 100 m2 na potrzeby użytkowników ścieżek rowerowych (zostawienie samochodów, postawienie budki na przechowanie  rowerów, które uczestnicy tych tras mogliby tu wypożyczać).
	
		\textbf{Opis przedsięwzięcia według  pomysłu autorów niniejszej pracy}
		
		Projekt dotyczy otwarcia wypożyczalni rowerów na terenie regionu przemyskiego. Wypożyczalnia będzie funkcjonowała tuż przy nowo powstałej sieci ścieżek rowerowych. Wybór wynika z zapotrzebowania na tego rodzaju aktywny wypoczynek dla osób miejscowych i przybyłych z zewnątrz. Chcemy aby nasza oferta była skierowana do ludzi na każdej płaszczyźnie wiekowej, dla grup osób indywidualnych, jak i całych rodzin.
		
		\textbf{Profil przedsiębiorstwa}\\
		Informacje o przedsiębiorstwie (nazwa, siedziba, status, bank):\\
		„Rowerek - Anna Kowalska”,\\
		ul. Kwiatowa 2/4\\
		32-342 Wyszatyce.\\
		Strona internetowa: www.rowerek.pl\\
		Jednoosobowa działalność gospodarcza\\
		Ta forma prawna umożliwia:
		
		......................
		
		...............
		
		................\\
		Bank nr konta: Bank PEKAO S.A. 	I O/Przemyśl\\
		Ul. Fiołkowa 37\\
		Nr konta 10000381240279918125200012.\\
		Nr identyfikacji podatnika podatku VAT – XXX-XXX-XX-XX.
		
		\subsection{Działalność przedsiębiorstwa}

		Właściciel spółki posiadają kapitał własny z rodzinnych oszczędności.
		Dodatkowo firma „Rowerek - Anna Kowalska" zaciągnęła kredyt w wysokości 5 000 zł. 
		
		Całkowity kapitał przeznaczony zostanie na:
		- zakup 22 szt. rowerów	-  						4 400  zł
		- dzierżawa parkingu 100 m2  (miesięcznie)  	-  			    244 zł
		- wiata do przechowywania rowerów	-				    500 zł
		(złożenie - praca własna)
		- 2 szt. Kabiny WC (od sponsora)	-        			                              0 zł
		- czynności formalno-prawne	-     					    300 zł
		- reklama lokalu „OSTOJA"	-     					    300 zł
		ulotki, „blackboardy"
		- przyczepa campingowa (dzierżawa)	-     				    350 zł
		- wyposażenie (chłodziarka, czajnik	-     				    600 zł
		elektryczny, grill, kuchenka, używane)
		- 4 stoliki i 16 krzeseł (reklama Coca-Cola	-         			        0 zł
		otrzymane od sponsora)
		- zakup pierwszej partii artykułów  spożywczych 	-     		    700 zł
		
		Koszty razem
		- 8 911,49  zł
		Przychody
		
		- pozostawienie samochodu na parkingu
		2,00 zł
		- wynajęcie roweru (dorosły)
		6,00 zł
		- wynajęcie roweru (dziecko)
		4,00 zł
		- umycie samochodu
		5,00 zł
		- sprzedaż map
		4,00 zł
		- sprzedaż informatorów
		3,00 zł
		- posiłki:
		
		zupa (kapuśniak, ogórkowa)
		
		3,50 zł
		bigos „Myśliwski"
		
		4,50 zł
		pierogi
		
		3,50 zł
		gofry z owocami i bitą śmietaną
		
		3,50 zł
		żurek
		
		4,00 zł
		flaczki
		
		5,00 zł
		zestaw „Mix”  ( frytki, mintaj panierowany, surówka)
		
		7,50 zł
		zapiekanki
		
		3,50 zł
		frytki
		
		2,50 zł
		hamburgery
		
		4,00 zł
		- napoje ciepłe
		
		herbata z cytryną
		
		1,50 zł
		kawa
		
		2,50 zł
		- napoje zimne
		
		sok (pomarańczowy, bananowy, brzoskwiniowy)
		
		2,00 zł
		coca cola
		
		2,50 zł
		piwo 0,5 1
		
		3,50 zł
		- lody (porcja)
		2,00 zł
		
		\textbf{Wynajęcie parkingu}\\
		„Rowerek” w celu otwarcia wypożyczalni rowerów Firma wynajmie powierzchnię asfaltową – 100 m2. Umowa wynajmu pomiędzy firmą „Rowerek” a właścicielem pływalni przy ul. Sportowej 18
		
		\textbf{Wypożyczenie przyczepy campingowej}\\
		Przyczepa będzie służyła do sporządzania posiłków regionalnych, kanapek, grillowanych kiełbasek, parzenia kawy, herbaty oraz sprzedaży napojów chłodzących.
		
		Przewidywane umiejscowienie przyczepy campingowej – na szlaku ścieżki rowerowej w połowie trasy rowerowej do Leszna. Posiłki będzie można zjeść przy stolikach pod zadaszeniem w chłodnym i czystym miejscu w otoczeniu przyrody wśród śpiewających ptaków.
		
		Zastawy obiadowe oraz sztućce będą przyozdobione ludowymi motywami ustawiane na serwetach haftowanych przez tutejsze gospodynie.
		
		\textbf{Produkt}\\
		Firma „Rowerek” oferować będzie szeroką gamę:
		\begin{itemize}
			\item dań gorących typowych dla kuchni regionalnej (wyszczególnione w karcie dań), tj. knedle z kapusty włoskiej, galaretkę z warzyw, rolmopsy, bigos „myśliwski”, krupnik (zupa zwana polską, gotowana na rosole wołowym),  pierogi, naleśniki  z serem, żurek, flaczki, zestaw „mix”, barszcz swojski, smalec ze skwarkami i ogórkiem kiszonym, dania z grilla;
			\item napoje alkoholowe – piwo „Leżajsk”,
			\item napoje bezalkoholowe: kawa, herbata, soki, napoje,
			\item lody z owocami, krem agrestowy.
		\end{itemize}
		Towary długoterminowe zakupywane będą w hurtowniach, a krótkoterminowe u zaprzyjaźnionych dostawców prowadzących sprzedaż zbiorowego żywienia (zbiorówek, kantyn, barów).

		\subsection{Forma prawna przedsiębiorstwa}

		\textbf{Procedura uruchomienia własnej działalności}\\
		Złożenie wniosku do Urzędu Miasta i Gminy o wpis do ewidencji działalności gospodarczych.
		Wniosek musi zawierać:
		\begin{enumerate}
			\item oznaczenie przedsiębiorcy oraz jego numer ewidencyjny – PESEL,
			\item oznaczenie miejsca zamieszkania i adres przedsiębiorcy,
			\item określenie podmiotu wykonywanej działalności gospodarczej zgodnie z Polską Klasyfikacją Działalności (PKD),
			\item wskazanie daty rozpoczęcia działalności gospodarczej.
		\end{enumerate}
		Obecnie w jednym okienku możemy:
		\begin{enumerate}
			\item dokonać wpisu do ewidencji działalności, czyli zarejestrować działalność gospodarczą,
			\item złożyć wniosek (formularz RG-1) o nadanie numeru statystycznego REGON.
			\item Zgłosić wniosek o nadanie numeru NIP (formularz NIP-1)
		\end{enumerate}
		Ostatnim zadaniem przedsiębiorcy jest wywieszenie szyldu, który zawiera:
		\begin{itemize}
			\item nazwę firmy,
			\item nazwisko przedsiębiorcy,
			\item adres działalności gospodarczej.
		\end{itemize}
	
		Oferta firmy „Rowerek” skierowana jest do szerokiego grona konsumentów, jakimi są młodzi i starsi ludzie preferujący wypoczynek na świeżym powietrzu, goście z krajów UE, ale także mieszkańcy gminy. Specyficzna atmosfera jaką chcemy stworzyć ma zaznajomić i przybliżyć kulturę i zwyczaje naszego regionu.
		
		Ważną rzeczą jest to, że firma „Rowerek” powstaje na nowo otwartych możliwościach, gdzie do tej pory nie było tego typu propozycji wypoczynku. Wynika z tego, że konkurencja jest niewielka i co za tym idzie, możliwość pozyskania klientów jest duża. Możemy stwierdzić, że jesteśmy bezkonkurencyjni. Miejsce rozdawania ulotek, udzielania informacji będą dokładnie rozpatrzone. Na pewno jednym ze sposobów będzie skorzystanie z lokalnej i regionalnej prasy: „Sztafeta”, Biuletyn Miejski miasta Przemyśl oraz dzienniki Super Nowości i Kurier Powiatowy.
		
		Ulotki oraz logo firmy zaprojektuje właścicielka Anna Kowalska.
		
		\subsection{Przeprowadzone postępowanie przygotowawcze w celu założenia przedsiębiorstwa}

		\textbf{Możliwości i szanse}\\
		Ta oferta skierowana jest do całego społeczeństwa, zarówno zamożnych jak i ubogich klientów, miejscowych i zagranicznych. Dajemy możliwość oderwania się od zgiełku zatłoczonych miast i warkotu jeżdżących samochodów. To czysty kontakt z naturą wśród pachnących łąk i lasów.
		
		\textbf{Podstawowe zagrożenia}\\
		Brak infrastruktury wzdłuż ścieżek rowerowych tj. kiosków z napojami chłodzącymi, z regionalną kuchnią, brak drogowskazów określających kierunek jazdy, punktów informacyjnych o miejscowych zabytkach, pamiątkach przyrody, czy ciekawych wydarzeniach. To spowoduje skrócenie czasu obcowania z przyrodą, gdyż głód i zmęczenie pokieruje klientów w drogę powrotną. W regionie nie mamy konkurencji, gdyż będziemy pionierami w tej dziedzinie.
		
		\textbf{Analiza otoczenia konkurencyjnego}\\
		Z przeprowadzonej ankiety wynika, że wiele osób wyjeżdża na dłuższe i krótsze weekendy do pobliskiego Rezerwatu Przyrody Starzawa nad zalew, na pobliską Ukrainę tylko dlatego, że nie ma na miejscu żadnej możliwości skorzystania z wypoczynku na świeżym powietrzu, a więc wielu z nich jest zainteresowanych taką formą wypoczynku, jaką będzie się zajmowała firma „Rowerek”.
				
		\textbf{Analiza SWOT}\\
		Na podstawie przeprowadzonych badań zostały określone mocne i słabe strony działalności oraz szanse i zagrożenia w otoczeniu firmy, będące podstawą do poznania charakterystycznych cech firmy, jej możliwości rozwoju oraz pozycji w stosunku do konkurencji w sektorze.
		
		\textbf{Mocne strony}
		\begin{itemize}
			\item Jedyna w swoim rodzaju (brak konkurencji).
			\item Skierowana do zamożnych i mniej zamożnych klientów.
			\item Dostosowana do potrzeb osób starszych i rodzin z małymi dziećmi.
			\item Wysoki poziom kapitału własnego.
			\item Łatwy dostęp do dostawców regionalnych (pamiątki, potrawy).
			\item Samozatrudnienie.
		\end{itemize}
		
		\textbf{Słabe strony}
		\begin{itemize}
			\item Brak infrastruktury wzdłuż ścieżek rowerowych.
			\item Niedostateczny przepływ informacji dotyczący potrzeb klientów.
			\item Brak jednolitej strategii rozwoju regionu (spójności działań ze strony władz miasta i gmin powiatu Przemyśl).
			\item Sezonowość.
			 Warunki pogodowe.
		\end{itemize}
	
		\textbf{Szanse}
		\begin{itemize}
			\item Napływ obcokrajowców i turystów z różnych stron Polski.
			\item Korzystanie z funduszy strukturalnych przeznaczonych na rozwój regionu.
			\item Wzrost zamożności społeczeństwa.
			\item Możliwość rozwoju spółki – wzrost, zapotrzebowania na imprezy wysokiej jakości.
			\item Rozwój sieci hotelarsko – gastronomicznej.
			\item Duża grupa lojalnych klientów.
		\end{itemize}
		\textbf{Zagrożenia}
		\begin{itemize}
			\item Rosnąca liczba poważnych, aktywnie działających na rynku konkurentów.
			\item Coraz bardziej atrakcyjna oferta firm konkurencyjnych.
			\item Niż demograficzny.
			\item Spadek realnej wartości złotówki.
		\end{itemize}

	\section{Rejestracja firmy}

		\subsection{Instytucje do których należy się udać (i co tam załatwić)}
		\subsection{Jakie druki wypełnić w tych instytucjach?}
		\subsection{Jakie dokumenty należy posiadać?}
		\subsection{Wypełnić konieczne formularze i druki}

	\section{Formy opodatkowania przedsiębiorstwa}

		\subsection{Dopuszczalne formy opodatkowania naszej firmy w Polsce}
		\subsection{Wybrana forma opodatkowania: charakterystyka i dlaczego}

	\section{Zatrudnianie pracowników}

		\textbf{Właściciel}\\
		Właścicielką dziłalności „Rowerek" jest Anna Kowalska.\\
		Anna Kowalska – autorka niniejszej pracy ukończy w 2005 roku Politechnikę Rzeszowską. \\
		Zawsze chciałam prowadzić własną działalność gospodarczą. Dlatego też zajmę się stroną kreatywności firmy i księgowością. W pierwszym etapie będę bacznie obserwować klientów i ich wymagania, aby podaż ofert dostosować do potrzeb.
		
		\textbf{Pracownicy}
		\begin{itemize}
		\item Właścicielka to także pracownik.
		\item Serwisant - zajmie się sprawami napraw i wypożyczania rowerów.
		\item Kucharka - będzie przygotowywała posiłki regionalne, sprzedawała herbatę, kawę, napoje chłodzące, lody.
		\item Sprzedawca - sprzedaż biletów na ścieżki rowerowe.
		\item Sprzedawca - sprzedaż posiłków.
		\end{itemize}
	
		\textbf{PLAN FINANSOWY}\\
		\textbf{Koszty}\\
		Miesięczny koszt zatrudnienia (brutto) oraz składki ZUS: 
		- kucharz x 1:
		849 zł + 18,71% = 1007,85 zł,
		-	właściciel (minimalna podstawa składek ZUS dla osób prowadzą
		cych działalność):
		1 361,96 zł x 18,71% = 254,82 zł,
		-	właściciel (minimalna podstawa składek ZUS dla osób prowadzą
		cych działalność:
		1 361,96 zł x 18,71% = 254,82 zł. 
		
		Koszt wynagrodzenia ogółem:		  1 517,49 zł.
		
		\subsection{Procedura zatrudniania pracowników}
		\subsection{Zawarcie umowy o dzieło, umowy zlecenie, umowy o pracę}

	\section{Wnioski} 

		\subsection{Zalety i wady wybranej formy prawnej przedsiębiorstwa}
		\subsection{Trudności napotkane w trakcie rejestracji firmy}
		\subsection{Co przyspieszyłoby proces rejestracji firmy?}
		\subsection{Inne wnioski}
		
		Opracowany biznesplan jest szansą dla rozwoju firmy „Rowerek”, a jego realizacja pozwoli właścicielce spełnić marzenie o własnej działalności gospodarczej. Z analizy rachunku zysków i strat wynika, że uruchomienie takiej działalności pozwoli na zapłacenie bieżących rachunków, opłacenie składek na ubezpieczenie społeczne, wypłacenie wynagrodzenia oraz spłaty miesięcznej raty zaciągniętego kredytu. Prognozy finansowe są obiecujące dla rozwoju firmy, a osiągnięte w przyszłości zyski zapewnią mocną pozycję przedsiębiorstwa w tej branży i pozwolą w inwestowanie w jego rozwój.
		
		Przedstawiony zarys pomysłu na uruchomienie jedoosobowej działalności gospodarczej z wykorzystaniem walorów turystycznych i kulturowych Przemyśla i okolic, jest przykładem innowacyjnego działania, przejawem poszukiwania pomysłu na samozatrudnienie. Jest to niewątpliwie jeden z pomysłów działalności gospodarczej na zasadach samozatrudnienia, a nie oczekiwania tylko na pracę u innych pracodawców.
		
		Moją intencją było w tym względzie pokazanie sposobu myślenia ludzi, którzy nie mogąc  znaleźć pracy u innych mogą sami szukać inspiracji w otaczających ich środowisku turystycznym, kulturalnym, społecznym i ekonomicznym w celu stworzenia sobie oraz na początek swoim najbliższym zajęcia zarobkowego.
		
		Moim zdaniem należy zaczynać zawsze od małej przedsiębiorczości, od niewielkiego pomysłu traktowanego jako poligon doświadczalny własnych umiejętności i predyspozycji. Przecież większość współczesnych średnich i wielkich firm, też kiedyś była małymi.

\end{document}