\documentclass{sprawozdanie-agh}

\usepackage[utf8]{inputenc}
\usepackage{listings}
\usepackage{pdfpages}
\usepackage{float}
\usepackage{anyfontsize}
\usepackage{graphicx}
 
\makeatletter 

\begin{document}   

	\przedmiot{Aspekty prawne i organizacja przedsiębiorstwa}
	\tytul{Jednoosobowa działalność gospodarcza}
	\podtytul{„Szukanie dzieci”}
	\kierunek{Informatyka, III rok, 2018/2019}
	\autor{Agnieszka Zadworny, Maciej Bielech, Tomasz Pęcak, Piotr Morawiecki}
	\data{Kraków, 13 grudnia 2018}

	\stronatytulowa{}

	\section{Informacje wstępne}

		\subsection{Cel i okoliczności powstania firmy}
		\subsection{Działalność przedsiębiorstwa}
		\subsection{Forma prawna przedsiębiorstwa}
		\subsection{Przeprowadzone postępowanie przygotowawcze w celu założenia przedsiębiorstwa}

	\section{Rejestracja firmy}

		\subsection{Instytucje do których należy się udać (i co tam załatwić)}
		\subsection{Jakie druki wypełnić w tych instytucjach?}
		\subsection{Jakie dokumenty należy posiadać?}
		\subsection{Wypełnić konieczne formularze i druki}

	\section{Formy opodatkowania przedsiębiorstwa}

		\subsection{Dopuszczalne formy opodatkowania naszej firmy w Polsce}
		\subsection{Wybrana forma opodatkowania: charakterystyka i dlaczego}

	\section{Zatrudnianie pracowników}

		\subsection{Procedura zatrudniania pracowników}
		\subsection{Zawarcie umowy o dzieło, umowy zlecenie, umowy o pracę}

	\section{Wnioski} 

		\subsection{Zalety i wady wybranej formy prawnej przedsiębiorstwa}
		\subsection{Trudności napotkane w trakcie rejestracji firmy}
		\subsection{Co przyspieszyłoby proces rejestracji firmy?}
		\subsection{Inne wnioski}

\end{document}